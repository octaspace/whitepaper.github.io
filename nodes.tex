\section{Nodes}

In the nutshell node is a Linux machine with special software installed.

This software is called \textbf{ORC}, using it \textbf{OCTA CORE} is able to establish secure communication channel to the node.

Communication between \textbf{OCTA CORE} and \textbf{ORC} is doing in RPC\cite{rpc} like manner.

Secure channel is implemented using HTTPS\cite{https} protocol with validating each request using security token.

\begin{figure}[H]
    \centering
    \includegraphics[scale=0.5]{core-orc-channel}
    \caption{Secure communication channel}
\end{figure}

\textbf{ORC} is responsible for the following activities:

\begin{itemize}
    \item Detect installed hardware: CPU, GPUs, RAM, volume of disk storage
    \item Collect metrics about hardware usage, like free/used disk/ram space, CPU and GPU load, temperature and fans speed
    \item Manage Docker\cite{docker} containers and Firecracker\cite{firecracker} microVMs
\end{itemize}

There are two types of nodes: \textbf{blockchain} and \textbf{service}

The first one is responsible for support of OCTA Layer 1 blockchain by running network node software.
As a result such nodes making network more stable, distributed, more latency fair and speed up the synchronization.

Service node provides resources we used to implement services for the end users.

\subsection{Hardware and software requirements}

To cover a big range of supported hardware node can be installed on any machines with x86\_64\cite{x86_64} or ARM\cite{arm} architecture.

Hardware must meet minimum system requirements: 1 CPU, 1 Gb of RAM, 10Gb of free disk space.

Depending on the purpose of the node, the requirements for hardware may change.

For example, for a node that provides only VPN, it is enough to meet the minimum requirements.
For nodes that are designed to perform tasks related to AI/ML, it is important to have powerful GPUs connected though fast PCIe interface and enough disk storage.

As for GPU, both NVIDIA and AMD GPUs supported as well.

To determine the performance of a node, the following measurements are performing:

\begin{itemize}
    \item Network upload/download speed
    \item Disk write speed
    \item GPU performance using AI benchmark (only for NVIDIA)
\end{itemize}

These perfomance metrics helps users to choose the hardware they need to for their tasks.

Node software can be installed on any OS Linux distribution but we focusing on Ubuntu LTS or Debian as a primary systems.

\subsection{Security}

To eliminate possible security risks for users using \textbf{ORC} on their machines we apply the following rules and guidelines:

\begin{itemize}
    \item \textbf{ORC} is open sourced, this may possible to make audit by other people that software does not have malicious code
    \item Keep code base of \textbf{ORC} is small as possible for ease auditing
    \item Node software run under non privilaged user and don't have any permissions which not need for operation
\end{itemize}

\subsection{Verification}

Ensuring that your node infrastructure operates smoothly and reliably is crucial for providing high-quality services to end-users.

Every new node joined to cloud must be verified and confirmed that it met requirements and able to provide service.

From time to time, we conduct re-verification checks on verified nodes, so it's essential to keep an eye on the status of your nodes to avoid them being changed to unverified.

What exactly checking to be sure what node is properly configured:

\begin{itemize}
    \item Met hardware requirement
    \item System clock is synchronized
    \item All necessary network ports are open
    \item GPU driver is correctly installed
\end{itemize}

The list of checks will be increasing in future.

The following restrictions are applied for the unverified nodes:

\begin{itemize}
    \item Nodes can't provide services
    \item Nodes can't participate in \hyperref[sec:staking]{staking}
\end{itemize}
