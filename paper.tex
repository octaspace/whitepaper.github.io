\documentclass[a4paper]{article}

\usepackage[utf8]{inputenc}
\usepackage[english]{babel}

\usepackage{hyperref}
\hypersetup{
    colorlinks=true,
    linkcolor=blue,
    filecolor=magenta,
    urlcolor=cyan,
    citecolor=blue,
    pdftitle={OctaSpace White Paper}
}
\usepackage{graphicx}
\usepackage[parfill]{parskip}
\usepackage{minted}

\graphicspath{ {./img/} }
\usepackage{geometry}
\geometry{
    a4paper,
    total={170mm,257mm},
    left=20mm,
    top=20mm,
}

\begin{document}

\title{OctaSpace White Paper}

\author{OctaSpace Team}

\maketitle

\tableofcontents
\newpage

\begin{abstract}
OctaSpace is a network that connects the owners of spare compute to the users of compute-heavy applications.

The platform aims to be as easy to use as centralized services and largely scalable to millions of computers and users.

The project's objective is to provide a functional and transparent way to perform computations in a distributed environment to store, process, and serve any amounts of data.

The nature of the computations may vary from machine learning, simulations, rendering, or even gaming, any task that requires massive computational power.
\end{abstract}
\newpage

\section{Introduction}

Octa.Space project is a distributed computing platform which aims to provide the transparent and easy-to-use mechanismes and interfaces
to solve the tasks by utilizing CPU and GPU resources of the nodes connected around the world.

A distributed system is a collection of autonomous computing elements
that appears to its users as a single coherent system.\cite{distributedsystems}
It's very important to have the system with unified interfaces and coherent behaviour which provides a single entry point for the end users interaction.

The project goals are:

\begin{itemize}
    \item Provide easy to use interface for describing and executing the tasks
    \item Control and planning of the consumed resources
    \item Provide playground for experiments with task distribution
    \item Utilize the power of blockchain technology to make most things transparent, including payments, tasks distribution and assigments
\end{itemize}

The sort of tasks the project is focused on:

\begin{itemize}
    \item Machine Learning
    \item CGI rendering
    \item Digital image processing
    \item Other tasks/fields which require massive compuation power to be solved
\end{itemize}

\newpage
\section{Architecture}

The following picture shows the core components and interfaces

\begin{figure}[h]
    \centering
    \includegraphics[width=\textwidth]{octa-arch}
    \caption{High Level Architecture}
\end{figure}

\subsection{Blockchain}

Utilizing Pirl's 51 percent attack proof and hybrid PoW and PoS Chain, transactions are computationally and financially impractical to reverse. The backbone of the blockchain is built using go-ethereum technologies, resulting in a feature-rich chain complete with oracles, state channels, and smart contracts.

The PoW/PoS system uses both PoW and PoS mechanisms simultaneously to provide maximum security against 51 percent attacks while minimizing the risk of centralization. By combining PoW and PoS, the network is secured by both computational power and the stake of validators, offering a more robust and energy-efficient way of maintaining the integrity of the blockchain while also significantly reducing the risk of double spend attacks. This blockchain is responsible for securing two layers.

\newpage

\subsection{Layer 1 network}

\textbf{OCTA Layer 1} is PoW\cite{pow} blockchain network is used for the frontend user financial operations using the native coin OCTA.

Network based on go-ethereum\cite{go-ethereum} codebase with the following specification:

\begin{itemize}
    \item Block time is 15 seconds
    \item Total supply is unlimited\footnote{Total supply will be reviewed after Mahasim fork}
    \item Block reward and halving implemented according to \hyperref[sec:mp]{Monenary policy}
    \item PirlGuard is used as protection mechanism from 51\% attack
    \item Transaction fee is 21 Gwei
\end{itemize}

Fair start of the network without premine and with genesis difficulty in 100Gh to prevent instant reward.

\subsection{Layer 2 network}

This is a side chain for Layer 1 network, implemented as PoA\cite{poa} network with a set of validators.

Used for handling internal transactions for the node services its lightning-fast performance and its seamlessly handling of high-frequency, high-usage operations,
resulting in reduced operational costs, speedy transactions and a massive boost in charging operations throughput.

\subsection{Blockchain gateways}

These gateways provides unified API for \textbf{OCTA CORE} layer to give it ability work with both blockchain networks.

This API is private and not accessible outside.

\subsection{OCTA CORE and Application Layer}
The engine of our system, the compute rental manager, seamlessly handles all requests for compute rentals and communicates with the nodes and user applications to make it effortless for users to access and use the resources of connected nodes.

This layer is designed to be user-friendly, so even those without technical expertise can take advantage of the available resources with ease.

Its the core engine of our system, it's responsible for the following operations:

\begin{itemize}
    \item Communicates, monitoring and low level interaction with nodes
    \item Handle requests for computing resources
    \item Provide interface for creating services on a top of resources provided by nodes
    \item Services usage charging and billing operations
    \item Provide API for automation or integration with third party systems
    \item Fraud control
    \item Statistic and telemetry of system usage
\end{itemize}

\subsection{Resource Layer}
While the Octa Chain is very capable at processing large amounts of transactions the real aim of the project is to provide real world applications and to bridge the gap of computational owners and computational users. To accomplish this we’ve built Octa nodes

This layer consist of hardware(nodes) connected to the OctaSpace cloud.

Nodes are the foundation of our compute and services marketplace, providing the necessary computational power to meet the demands of tasks.

These computers are equipped with a blend of CPUs, GPUs, memory, and disk space that allows them to handle distributed workloads with ease.

The nodes are connected to the OctaSpace, which enables them to seamlessly work together to deliver optimal performance and efficiency.

The tasks and services may vary, well equipped machines with powerful GPU may performs AI/ML tasks,
common machines may acts as VPN gateway, provide disk storage for services like file sharing or host applications deployed by users.

\subsection{API and UI}

To work with system the following interfaces available:

\begin{itemize}
    \item Web applicaton with user-friendly interface: \url{https://cube.octa.space}
    \item RESTful API
    \item \textbf{octactl} command-line utility that provide user friendly interface to RESTful API
\end{itemize}

\newpage
\section{Nodes}

A node is a Linux machine with special software installed, called ORC, which enables OCTA CORE to establish a secure communication channel with the node.

The communication between \textbf{OCTA CORE} and \textbf{ORC} is done in an RPC\cite{rpc}-like manner,
and a secure channel is implemented using the HTTPS\cite{https} protocol, with each request being validated using a security token.

\begin{figure}[H]
    \centering
    \includegraphics[scale=0.5]{core-orc-channel}
    \caption{Secure communication channel}
\end{figure}

\textbf{ORC} is responsible for the following activities:

\begin{itemize}
    \item Detecting installed hardware suchas CPU, GPUs, RAM, volume of disk storage
    \item Collecting metrics about hardware usage, suhc as free/used disk/ram space, CPU and GPU load, temperature and fans speed
    \item Manage Docker\cite{docker} containers and Firecracker\cite{firecracker} microVMs
\end{itemize}

There are two types of nodes: \textbf{blockchain} and \textbf{service}

The blockchain node supports the OCTA Layer 1 blockchain by running network node software, which makes the network more stable, distributed, more latency fair, and speeds up synchronization.

The service node provides resources that are used to implement services for end-users.

\subsection{Hardware and software requirements}

To cover a wide range of supported hardware, node software can be installed on any machine with x86\_64\cite{x86_64} or ARM\cite{arm} architecture.

In order to run the node, the hardware must meet the minimum system requirements, which are as follows: 1 CPU, 1 Gb of RAM, and 10Gb of free disk space.

However, the requirements for hardware may vary depending on the intended purpose of the node. For instance, a node that provides only VPN services may only need to meet the minimum requirements. Conversely, nodes that are designed to perform AI/ML tasks will require powerful GPUs connected with a high bandwidth PCIe interface, as well as ample disk storage.

It's worth noting that both NVIDIA and AMD GPUs are supported by the system.

To determine the performance of a node, the following measurements are taken:

\begin{itemize}
    \item Network upload/download speed
    \item Disk write speed
    \item GPU performance using AI benchmark (only for NVIDIA)
\end{itemize}

These performance metrics help users to choose the hardware they need for their tasks.

The node software can be installed on any Linux distribution; however, we primarily focus on Ubuntu LTS or Debian as the recommended operating systems.

Windows Subsystem for Linux (WSL) has limited support.

\subsection{Security}

Security is of utmost importance to us, and we take various measures to eliminate possible security risks for users running our software ORC on their machines.

We follow a set of rules and guidelines to ensure the security of our system, which includes:

\begin{itemize}
    \item Open-sourcing \textbf{ORC}, which allows for audits by other people to ensure that the software does not have any malicious code
    \item Keeping code base of \textbf{ORC} as small as possible for ease of auditing
    \item Running node software under a non-privileged user and not granting any permissions that are not needed for its operation
    \item Regular software updates to ensure that the latest security patches are applied. Along with comprehensive testing of software to detect and address any potential security issues
\end{itemize}

These measures are in place to ensure that our users can have peace of mind when using our system, and their data and resources are secure.

We take the security of our users very seriously and are continuously working to improve our system's security features.

\subsection{Verification}

Verification is critical to ensuring that the node infrastructure operates smoothly and reliably, which is essential for providing high-quality services to end-users.
Every new node that joins the OctaSpace cloud must be verified and confirmed to meet the necessary requirements to provide services.

Periodic re-verification checks are conducted on verified nodes. It is therefore essential to monitor the status of your nodes to avoid them being changed to an unverified status.

The checks performed to ensure that the node is properly configured may include, but are not limited to:

\begin{itemize}
    \item Meeting minimum hardware requirements
    \item Synchronized system clock
    \item All necessary network ports are open
    \item Correct installation of the GPU driver
\end{itemize}

The list of checks will be expanding in the future to ensure even greater accuracy and reliability.

The following restrictions are applied for the unverified nodes:

\begin{itemize}
    \item Unable to provide services
    \item Unable to participate in staking \hyperref[sec:staking]{staking}
\end{itemize}

\newpage
\section{Services}

Services are what we build and what the project offers to end users.

These are things that can be used every day to solve problems from different areas, from complex calculations to simple file sharing between friends.

Below are some of the services that we have implemented or plan to implement in the near future.

In our opinion, these services that may be of interest to users and will find applications for every day.

\subsection{GPU marketplace}

GPU rental

\subsection{VPN}

VPN

\subsection{HashCache}

HashCache

\subsection{Instant File Sharing}

Instant File Sharing

\newpage
\section{Monetary Policy}
\label{sec:mp}

\textbf{OCTA} is the primary payment instrument and the currency used to pay for the services provided by OctaSpace, as well as rewards to node owners and dividend payments for OCTA holders.

In order to maintain stable inflation levels, OctaSpace implements a finite monetary policy.

The policy consists of a series of eras, each with a set duration and total coin supply. The block reward for each era is gradually reduced, as shown in Table 1, to decrease inflation and ensure a controlled supply of OCTA tokens.


In summary, the finite monetary policy ensures that the supply of OCTA tokens remains stable and controlled, while also providing rewards to those who contribute to the network through mining and staking. The gradual reduction of block rewards in each era helps to decrease inflation and maintain a healthy token economy for the long-term benefit of the OctaSpace community.

\begin{table}[h!]
\centering
\begin{tabular}{||c c c c c c||}
    \hline
        Era & Start block & Total & Miner & Staking & Dev \\ [0.5ex]

        \hline\hline
        Octa & 1 & 8 & 6.5 & 0 & 1.5 \\
        Arcturus & 650\_000 & 8 & 5 & 1.5 & 1.5 \\
        Oldenburg & 1\_000\_000 & 7.5 & 4 & 2 & 1.5 \\
        Zagami & 1\_500\_000 & 7 & 3.5 & 2.5 & 1 \\
        Springwater & 2\_000\_000 & 6.5 & 3 & 3 & 0.5 \\
        Polaris & 2\_500\_000 & 6 & 2.8 & 2.8 & 0.4 \\
        Mahasim & 3\_000\_000 & 5 & 2.3 & 2.3 & 0.4 \\
        Dnepr & 4\_500\_000 & 4 & 1.85 & 1.75 & 0.4 \\
        Blackeye & 6\_000\_000 & 2.5 & 1.2 & 1 & 0.3 \\
        Vega & 8\_000\_000 & 2.25 & 1.10 & 0.85 & 0.3 \\
        Triangulum & 10\_000\_000 & 2 & 1 & 0.7 & 0.3 \\[1ex]
    \hline

\end{tabular}
\caption{Reward distribution according to era}
\label{table:1}
\end{table}

This policy should decrease inflation by changing amount of block reward dependents of era.

\newpage

\begin{table}[h!]
\centering
\begin{tabular}{||c c c c c||}
    \hline
        Era & Start date & End date & Total coins & Duration \\ [0.5ex]

        \hline\hline
        Octa & Jun 19 2022 & Sep 26 2022 & 5\_200\_000 & $\approx$69 days \\
        Arcturus & Sep 26 2022 & Nov 11 2022 & 2\_800\_000 & $\approx$53 days \\
        Oldenburg & Nov 18 2022 & Feb 01 2023 & 3\_750\_000 & $\approx$75 days \\
        Zagami & Feb 01 2023 & Apr 16 2023 &  3\_500\_000 & $\approx$74 days \\
        Springwater & Apr 16 2023 & Jun 29 2023 & 3\_250\_000 & $\approx$74 days \\
        Polaris & Jun 29 2023 & Sep 12 2023 & 3\_000\_000, & $\approx$74 days \\
        Mahasim & Sep 12 2023 & Feb 08 2024 & 5\_000\_000 & $\approx$149 days \\
        Dnepr & Feb 08 2024 & Dec 03 2024 & 8\_000\_000 & $\approx$298 days \\
        Blackeye & Dec 03 2024 & Sep 27 2025 & 5\_000\_000 & $\approx$298 days \\
        Vega & Sep 27 2025 & Jul 23 2026 & 4\_500\_000 & $\approx$298 days \\
        Triangulum & Jul 23 2026 & May 18 2027 & 4\_000\_000 & $\approx$298 days \\[1ex]
    \hline

\end{tabular}
\caption{Approximate calculation of era timeline and reward distribution}
\label{table:1}
\end{table}

\begin{figure}[ht]
    \centering
    \includegraphics[width=\textwidth]{block_supply}
    \caption{Supply and Daily emissions}
\end{figure}

\newpage
\section{Staking}
\label{sec:staking}

To incentive interest of holding the staking mechanism is introduced.

Using it people who locked some amount of coins and run node is rewarded.

There are requirements which need met to activate staking:

\begin{itemize}
    \item Collateral - 100\_000 OCTA
    \item Node reliability for the last 30 days $\ge$ 75\%
    \item Node must be verified
\end{itemize}

To start staking it's necessary to have the address wallet with enough balance and link it to existing node.
Reward will be come to the wallet provided.

If system detect balance less than collateral staking will be disabled for such wallet-node pair for
several rounds.

\subsection{Reward distribution scheme}

The billing period is 1 week (every 7 days).

Let's assume that block time is always 15 seconds then the following amount of blocks would be mined:

\begin{itemize}
    \item 1 minute - 4
    \item 1 hour - 240
    \item 1 day - 5760
    \item 7 days - 40320
    \item 30 days - 172800
\end{itemize}

In this calculation we will use reward amount for Arcturus and Oldenburg eras where 1.5 OCTA from each block going to staking fund.

The total amount of coins mined for 1 week would be 60480 OCTA.

Let’s assume we have 10 nodes up and running, then the 60480 OCTA would be distributed across they in the following proportion:

\begin{itemize}
    \item 60\% (base reward) - 36240 OCTA, 3624 for each node
    \item 20\% (gpu reward) - 12080 OCTA, 1208 for each node which provide renting service with GPU
    \item 10\% (vpn reward) - 6040 OCTA, 604 for each node which provide VPN service
    \item 10\% (rent reward) - 6040 OCTA, 604 for each node which provide renting service
\end{itemize}

In case of no nodes for which needs to distribute 40\% (except base reward) the coins will be left in staking fund for the next round.

Most profitable variant is to have node with GPU and which provides both services.
Such node will be rewarded by 6040 OCTA per week and also additional payments for services usage.

Due to the current scheme, we can calculate minimal monthly ROI in the following way:

\[
    ROI = \frac{MonthReward / Collateral * 100}{N} = \frac{M}{100} * 60
\]

Where \textbf{MonthReward} is 172800 blocks multiply by staking reward according to current era.

\textbf{N} is amount of nodes participated in staking, for example 10.

Therefore the minimal ROI for Arcturus and Oldenburg eras will be:

\[
    ROI = \frac{172800 / 100000 * 100}{10} = \frac{17.28}{100} * 60 = 10.36\%
\]


\bibliography{paper}
\bibliographystyle{plain}

\end{document}
