\section{Services}

Services are what we build and what the project offers to end users.

These are things that can be used every day to solve problems from different areas, from complex calculations to simple file sharing between friends.

Below are some of the services that we have implemented or plan to implement in the near future.

In our opinion, these services that may be of interest to users and will find applications for every day.

\subsection{GPU marketplace}

This service provide ability to rent or rent out GPUs power.

This power may used to solve tasks in AI/ML, CGI rendering or any other fields where powerful GPUs is necessary.

Because we support both NVIDIA and AMD this increase the spectrum of tasks be solved.

Access to the rented instances is provided via SSH\cite{SSH} protocol.

Also it's possible to use Jupyter\cite{jupyter} and LiveBook\cite{livebook} systems for interactive access.

The rented GPU instances can be combined into a cluster, which makes it possible to run and develop distributed programs,
for example distributed training of ML models using TensorFlow\cite{tensorflow} or PyTorch\cite{pytorch}.

\subsection{VPN}

OCTA VPN\cite{VPN} offers a variety of key benefits to its users.
One of the main advantages is its ease of setup, which is made possible by utilizing non-modified and open-source software that is compatible with a wide range of platforms.

Additionally, users have the flexibility to choose from a variety of VPN technologies to suit their needs.
There are also no limitations on the number of devices that can be connected simultaneously.

To add to that, the billing model is pay-as-you-go, which means you are billed only for the amount of data you use.
This gives you complete control over your usage and costs.

Service don’t any logging of the user traffic or DNS requests.

VPN access is implemented using the following technologies:

\begin{itemize}
    \item WireGuard
    \item ShadowSocks
    \item OpenVPN
\end{itemize}

In the future more VPN types will be added, including some to bypass China's golden shield (Great Firewall of China).
\subsection{Remote Video Gaming}

Many have tried and failed with Stedia recently falling. All these services essentially rely on massive centralized server farms which inherently introduces huge latency issues resulting in bad gaming experience if your not located with 100km of the facility.

But with OctaSpace and enough nodes it may finally be possible for a remote gaming service to be widly used utilizing nodes across the globe allowing gamers to use share unused hardware. By connecting to say your neighbours unused pc would solve the major problem of latency in remote gaming services.

\subsection{HashCache}

HashCache is a Password Recovery service for performing password cracking operations on a large scale. It utilizes multiple nodes working together in a coordinated manner to speed up the cracking process. This is achieved by dividing the cracking workload among the nodes, allowing them to work in parallel on different parts of the task.

The system is designed to be highly configurable and can be optimized for different types of cracking operations, such as dictionary attacks, brute force attacks, and others.

\subsection{Distributed Rendering}

Another service example would be a distributed handbrake video encoding network. To allow content creators or studios to quickly render out massive edits economically.

\subsection{Instant File Sharing}

This service provide a easy way to upload a file, share a short link, and after it is downloaded, the file is completely deleted.

Key benefits:

\begin{itemize}
    \item All files are encrypted
    \item It's possible to set expiration (Time To Live), after such period file will be deleted
    \item Simple RESTful API
\end{itemize}
